\title{康普顿效应验证实验}
%\thanks{本文为2018级本科生顾周洲(学号\href{mailto:zzgu@pku.edu.cn}{1800011320})和陈天扬(学号\href{mailto:chentianyang@pku.edu.cn}{1800011327})在2019-2020学年进行的研究性实验工作的总结报告,作者按学号排序、不分先后.指导教师为张朝晖老师和王伟老师.}

\author{顾周洲}
%\email[邮箱:]{zhangzh@pku.edu.cn}
\affiliation{北京大学物理学院\\ 中国,北京,颐和园路5号\ 100871}
\collaboration{2021年 10 月}

\begin{abstract}
    本实验以\csAtom 为放射源,测定了其放射出的\SI{0.662}{MeV} $\gamma$射线被铝棒散射后的能量及相对微分散射截面关于散射角的分布。实验结果与康普顿散射的光子能量随散射角、以及Klein–Nishina 公式给出的散射截面随散射角变化关系符合得相当好,进一步分析对实验的误差给出了解释,从而验证了对康普顿效应的理论计算。


\end{abstract}